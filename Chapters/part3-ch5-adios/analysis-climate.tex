\noindent{\bf Climate research.}

Climate modeling applications address one of the most pressing issues faced by our society. Numerous
surveys show increasing trends in the average sea-surface and air temperatures. A promising method
of predicting the future effects of climate change is to run large-scale simulations of the global
climate many years into the future. In such simulations, climatologists discretize our entire globe,
setting up differential equations to model Earth's climate. These climate simulations can generate
several terabytes, or even petabytes, of raw data. Such data-intensive applications present a
tremendous set of challenges for computer centers that strive to accommodate, maintain, transfer,
and preserve data.

Large-scale computing centers that host such applications cannot afford to frequently stall on data
accesses and lose precious computing cycles. The I/O performance of scientific applications such as
climate models has a great impact on the completion of scientific simulation and post-mortem data
visualization and analysis. However, many applications have complex data characteristics that are
not well supported by existing parallel I/O libraries. For example, applications may generate large
number of small variables. While the data is large in total, each process only holds very small
amount of data for each variable. An example is one of the NASAs climate and weather models named
GEOS-5 (the Goddard Earth Observing System Model). A GEOS-5 simulation at a coarse resolution of
half-degree generates only 3.12GB data but which consists of 185 2-D variables and 80 3-D variables
at a time. The number of time steps can be configured to the order of thousands. It is challenging
to provide good I/O speed for both writing and reading where large amount of small I/O requests are
expected. Moreover, it is also quite common for data post-processing to examine data along time
dimension, e.g. observing the change of temperature within one hour. However, storage data layouts
currently common in use do not provide a good support for such access pattern.

An I/O technique that can significantly enhance the spatial and temporal resolution of these climate
and weather models can systematically lift the capability of climatologists and meteorologists in
addressing critical climate issues. An I/O method of ADIOS was specifically designed to address
these problems~\cite{ADIOS:Tian:msst13}. It leverages the spatial and temporal relationships between
variable data. Spatially it aggregates and merges data chunks of the same time steps so that fewer
processes write larger blocks. Moreover, the same variable across different time steps are merged
together with time as a new dimension, which again reduces the amount of I/O request for both
writing and reading. This strategy has provided two orders of magnitude of speedup compared to
simple writing (and reading) strategies.





