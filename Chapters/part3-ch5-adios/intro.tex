{\color {red}What is the purpose of the library?}

Scientists do not really care about I/O but about their scientific discovery process. Ideally, one would like to watch a simulation run in real time using some automatic or interactive visualization at a remote location. Ideally, one would like to store a large amount of data for post-processing; anything that might be interesting or useful for finding new information. Ideally, one would like to combine data from experimental facilities or observations with their simulation data to verify the simulation code or to setup a new experiment with simulation aid. The bottlenecks of the I/O system and I/O libraries, however, prevent the scientists to do just that. Instead, they have to focus on determining the limitations and design a scientific process around a limited amount of data that can be stored and around a limited bandwidth for moving the data from one place to the other. The primary goal of designing ADIOS was to create an I/O framework that enabled scientists to focus on creating their data driven scientific processes, writing their applications with a simple I/O programming interface, while the data movement is taken care of automatically, whether it is written to permanent storage, moved from the memory of one application to the memory of another, or across the network.

The Oak Ridge Leadership Facility (OLCF) has reached the No. 1 position on the Top 500 list of supercomputers twice in the past and also operates one of the largest Lustre-based parallel file system (10PB storage with 640 file system servers) along with the computing resources. The I/O limitations at the largest scale are the most painful. Applications, which are designed to scale computationally well, soon face the I/O as bottleneck. The problem is serious; many applications running above thirty-thousand cores achieve less then 1\% of the practically available\footnote{"practically" here means the number provided by testing the file system with I/O kernel codes as opposed to the "theoretically available" bandwidth published by a vendor.}  I/O bandwidth. The responsibility for the bad performance can be divided up between the application itself, the I/O library and the file system. 

At the large scale that applications are running at the leadership facilities, it is clear that none of the file-per-process or the single-shared-file writing strategy scales well. Aggregation between processes is required to avoid contention at the storage servers, along with local buffering of data in the memory to utilize the available I/O bandwidth by writing large blocks of data in bursts. However, the level of aggregation depends on the application size, available memory, the number of storage servers, the load on the storage system shared by other applications and so on. At the application level, it is very hard to create a flexible I/O routine that works well on all supercomputers the scientist has allocation time. The ones that try to do that on their own, end up with creating effectively a new I/O middleware to hide the complexity or end up with un-maintainable mess of a code.

The second goal in the design of ADIOS was to provide an I/O library, which took away the burden of implementing the I/O strategy at the application level, and which allowed for choosing the best performing I/O strategy depending on the parameters of an actual run and the target storage system. Each process in an application needs only to declare what data should be moved and when, just like with other I/O libraries, with the exception that the I/O strategy (e.g., aggregation and output file organization) is not implemented at this level. ADIOS takes care of the data movement using a selected I/O strategy, which by this way, can be changed between simulation runs without rebuilding the application code. 

