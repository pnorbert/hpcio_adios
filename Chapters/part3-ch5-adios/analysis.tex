\noindent{\bf Seismography.}
Recent advances in high-performance computing and numerical techniques have facilitated full 3D
simulations of global seismic wave propagation at high resolution. The spectral-element method (SEM)
has recently gained popularity in seismology~\cite{ADIOS:seismography:tromp2008spectral}, and now the SEM
is widely used for 3D global simulations. The Theoretical~\& Computational Seismology research group
at the Princeton University has developed the open-source SEM package
SPECFEM3D\_GLOBE~\cite{ADIOS:seismography:carrington2008high}. Their goal is to improve the models of
Earth's subsurface and kinematic representations of earthquakes.

There are about 6000 earthquakes recorded as of today, with several thousands of seismograms
available for each earthquake. About 50 iterations of a long processing pipeline for each earthquake
will be needed to achieve a highly accurate subsurface model of the whole Earth. In the pipeline,
thousands of seismograms (both observed and synthetic ones) are processed and produced at each step
for each earth quake while several earth quakes are processed concurrently. Observational data is
provided by the IRIS organization but it has to be processed (cleaned and aligned) for the
comparative analysis with the synthetic datasets. The current standard data format is a separate
binary file for each seismogram, resulting in millions of small input files and multiplies of that
number of files for intermediate synthetic data. Even the largest parallel file systems cannot
provide access to these many files with reasonable performance. Therefore, a new seismography data
format has been developed that stores all seismograms of an earth quake in a single ADIOS file, which
provides efficient access to all of them while reducing the total number of files by several orders
of magnitudes. This change alone allows for executing the processing pipeline on the Titan
supercomputer at the Oak Ridge Leadership Facility, using the allocated time efficiently, i.e., not
by mostly waiting on reading and writing the small files.
 
