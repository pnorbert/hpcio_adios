Application of data reduction techniques with ADIOS have shown to be effective in reducing the
bottleneck on I/O during simulation writes. These techniques fit well with the minimal communication
principle employed by ADIOS, where each process handles its own I/O. By applying compression
routines locally on every process, encoding costs are effectively minimized. This also enables
compression techniques to take advantage of similarity in data values within each process, typically
seen with spatio-temporal scientific datasets. For example, ISOBAR \cite{ADIOS:Schendel2012b}, an in situ
lossless compression routine specific for scientific data, demonstrated up to a 46\% reduction in
storage on datasets from simulations spanning various domains such as combustion (S3D), plasma
(XGC1, GTS) and astrophysics (FLASH). This technique coupled with ADIOS and with the addition of
interleaving allowed throughput gains proportional to the degree of data reduction.

While checkpoint/restart data such as particle data must be compressed losslessly, datasets that are
used for analysis and visualization, such as field data, can be compressed in a lossy fashion.
Unlike lossless compression, lossy compression techniques such as wavelets-based compression,
ISABELA \cite{ADIOS:Lakshminarasimhan2013a}, etc. provide a multi-fold reduction in storage sizes, trading
precision for reduced storage. Depending on the sensitivity of the analysis routines to errors
introduced by the compression process, the end-users can change the level of accuracy desired in the
configuration file used by ADIOS. The configuration file can also instruct ADIOS to employ different
compression routines for different variables and output groups (checkpoint/restart or analysis),
without having to change the application code.

