Most applications face the I/O bottleneck first when scaling up is the dumping of all data necessary to restart the code in case of failures or when a simulation consists of multiple consecutive job executions on machines with time-limited jobs. Many scientists hedge their bets against the probability of failure and skip writing checkpoint data entirely, wasting a lot of CPU hours in some of the runs but consistently saving time in every successful run. Writing checkpoint data represents two classes of I/O challenges: write 1) large or 2) small amount of data, stored in many variables, from every process of the application. In both cases, buffering helps to decrease the amount of individual write operations, thus avoiding I/O latency and utilizing the available I/O bandwidth. In both cases, aggregation is required to decrease the number of entities hitting on the file system at the same time. ADIOS consistently improved all large scale applications' checkpoint writing (and reading) performance by 10-100x.


