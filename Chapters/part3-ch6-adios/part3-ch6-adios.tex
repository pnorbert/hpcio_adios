%\chapterauthor{Norbert Podhorszki,  Scott Klasky,  Sriram Lakshminarasimhan,  Manish Parashar,  Karsten Schwan,  Yuan Tian,  Matthew Wolf}{ORNL, NCSU, RSU, GTech}

\chapter{ADIOS}
\label{part3-ch6-adios}
{\color {red}Please target 5-8 pages for this chapter (including figures+references). Readers should be able to appreciate the features of the I/O library, and why they might consider using it for their applications.}

%
% Motivation
%
\section{Motivation}
{\color {red}Scott}

{\color {red}What is the purpose of the library?}

Scientists do not really care about I/O but about their scientific discovery process. Ideally, one would like to watch a simulation run in real time using some automatic or interactive visualization at a remote location. Ideally, one would like to store a large amount of data for post-processing; anything that might be interesting or useful for finding new information. Ideally, one would like to combine data from experimental facilities or observations with their simulation data to verify the simulation code or to setup a new experiment with simulation aid. The bottlenecks of the I/O system and I/O libraries, however, prevent the scientists to do just that. Instead, they have to focus on determining the limitations and design a scientific process around a limited amount of data that can be stored and around a limited bandwidth for moving the data from one place to the other. The primary goal of designing ADIOS was to create an I/O framework that enabled scientists to focus on creating their data driven scientific processes, writing their applications with a simple I/O programming interface, while the data movement is taken care of automatically, whether it is written to permanent storage, moved from the memory of one application to the memory of another, or across the network.

The Oak Ridge Leadership Facility (OLCF) has reached the No. 1 position on the Top 500 list of supercomputers twice in the past and also operates one of the largest Lustre-based parallel file system (10PB storage with 640 file system servers) along with the computing resources. The I/O limitations at the largest scale are the most painful. Applications, which are designed to scale computationally well, soon face the I/O as bottleneck. The problem is serious; many applications running above thirty-thousand cores achieve less then 1\% of the practically available\footnote{"practically" here means the number provided by testing the file system with I/O kernel codes as opposed to the "theoretically available" bandwidth published by a vendor.}  I/O bandwidth. The responsibility for the bad performance can be divided up between the application itself, the I/O library and the file system. 

At the large scale that applications are running at the leadership facilities, it is clear that none of the file-per-process or the single-shared-file writing strategy scales well. Aggregation between processes is required to avoid contention at the storage servers, along with local buffering of data in the memory to utilize the available I/O bandwidth by writing large blocks of data in bursts. However, the level of aggregation depends on the application size, available memory, the number of storage servers, the load on the storage system shared by other applications and so on. At the application level, it is very hard to create a flexible I/O routine that works well on all supercomputers the scientist has allocation time. The ones that try to do that on their own, end up with creating effectively a new I/O middleware to hide the complexity or end up with un-maintainable mess of a code.

The second goal in the design of ADIOS was to provide an I/O library, which took away the burden of implementing the I/O strategy at the application level, and which allowed for choosing the best performing I/O strategy depending on the parameters of an actual run and the target storage system. Each process in an application needs only to declare what data should be moved and when, just like with other I/O libraries, with the exception that the I/O strategy (e.g., aggregation and output file organization) is not implemented at this level. ADIOS takes care of the data movement using a selected I/O strategy, which by this way, can be changed between simulation runs without rebuilding the application code. 




%
% Design/Architecture
%
\section{Design/Architecture}

{\color {red}Where is the library positioned in the I/O stack? What design constraints drove the architecture?}

ADIOS~\cite{ADIOS:Lofstead:ipdps09} sits on the top of the I/O stack as a user-level I/O library.
The write API of ADIOS is designed to be as close to the POSIX I/O calls as possible, but without a
fix specification about what happens in a "write" call. Datasets are "opened" and "closed" and
variables are "written" to them. In general, write calls only buffer the data content, and in the
close call, ADIOS performs burst writes to the file system. However, it depends on the I/O method
used in an actual run, what is done with the data.

The API lets the scientist express I/O in terms of variables in the code as with HDF5 or NetCDF. An
ADIOS dataset consists of typed, multi-dimensional arrays or simple scalars, with global dimensions
defined by scalar variables (similar to NetCDF dimensions) or simply by integer values. A writing
process also has to declare where its array fits in a global array using offsets and local
dimensions.

ADIOS methods implement different I/O strategies, like writing one file per process, or a single
share file, or aggregating data into a certain number of files; to push data into the memory of
another application using memory-to-memory transfers, or pull data from another application, and so
on. The actual method performing I/O can be selected at runtime in the application.

The file format designed for ADIOS, and used by most file-based I/O methods in ADIOS, provides a
self-describing data format, a log-based data organization and redundant metadata for resiliency and
performance. The log-based data organization allows for writing each process' output data into a
separate chunk of the file concurrently. In contrast with logically contiguous file formats where
the data in the (distributed) memory of the processes has to be reorganized to be stored on disk
according to the global, logical organization, this format eliminates i) communication among
processes when writing to reorder data and ii) seeking to multiple offsets in the file by a process
to write data interleaved with what is written by other processes.While local buffering by the
processes exploits the best available I/O bandwidth by streaming large, contiguous chunks to disk,
the destination format itself avoids the common bottlenecks that would hamper that performance. The
many processes writing to different offsets in a file or to different files avoid each other on a
parallel file system to the extent possible. In most cases, each process attempts to write to a
single stripe target to avoid the metadata server overhead of spanning storage targets. The reading
performance of this file format was shown to be generally advantageous as
well~\cite{ADIOS:Lofstead:hpdc11}.





%
% Deployment
%
\section{Deployment, Usage, applications}
{\color {red}Brief summary of where the library is deployed, who uses it, etc.

Note: you can choose to mention some performance numbers in the context of applications, but for the purpose of this book we would like you to avoid direct comparisons of your favorite library with other technologies.  }


%
%
\subsection{Checkpoint/restart}

Most applications face the I/O bottleneck first when scaling up is the dumping of all data necessary to restart the code in case of failures or when a simulation consists of multiple consecutive job executions on machines with time-limited jobs.  Writing checkpoint data represents two classes of I/O challenges: write 1) large or 2) small amount of data, stored in many variables, from every process of the application. In both cases, buffering helps to decrease the amount of individual write operations, thus avoiding I/O latency and utilizing the available I/O bandwidth. In both cases, aggregation is required to decrease the number of entities hitting on the file system at the same time. ADIOS consistently improved all large scale applications' checkpoint writing (and reading) performance by 10-100x and thus allowed for writing checkpoint data of the size of multiple terabytes at a time with acceptable I/O overhead. 

An application, however, does not need to run on tens of thousands of cores to experience I/O bottlenecks. An example application is industrial engineering where Computational Fluid Dynamics (CFD) simulations on billions of cells are necessary today to get new insights in turbulent behaviors. Fine/Turbo is a CFD code of Numeca International. The irregular topology of a specific simulation is partitioned into small regular 3D blocks. Its load balancing algorithm divides up the blocks and the variables on each block among the processes in a way that different variables in a block end up having different sets of processes assigned to them. The original sequential  I/O naturally could not scale up for thousands of cores. Collecting all data on a single core led to too much communication.  Attempts to write in parallel, while retaining each variable of each block organized on the disk as a contiguous block of data, led again to too much communication between processes because the merge of each variable on each block required a gathering operation from a different set of processes. 

ADIOS provided the optimal solution to the writing problem. Each process dealt with its own I/O, thus completely avoided cross-communication among the processes. The data on each process was buffered locally and then, because of the ADIOS-BP file format, it was written into a separate segment of the output file, that is, with one write request with a single seek in the output. The original sequential I/O took 2000 seconds to complete writing (a smallish) 18GB output of a 500 million cells simulation running on 1280 cores. ADIOS, with the single-shared-file approach completed the same I/O in 60 seconds. However, scaling the application up to 8k processes showed the limitations of the single file approach, and therefore aggregation with multiple files was finally used. With aggregation, the above example was completed in 12 seconds, and scalability of the code was maintained for larger simulations on 3.7 billion cells.



%
%
\subsection{Analysis}

\noindent{\bf Seismography.}
Recent advances in high-performance computing and numerical techniques have facilitated full 3D
simulations of global seismic wave propagation at high resolution. The spectral-element method (SEM)
has recently gained popularity in seismology~\cite{ADIOS:seismography:tromp2008spectral}, and now the SEM
is widely used for 3D global simulations. The Theoretical~\& Computational Seismology research group
at the Princeton University has developed the open-source SEM package
SPECFEM3D\_GLOBE~\cite{ADIOS:seismography:carrington2008high}. Their goal is to improve the models of
Earth's subsurface and kinematic representations of earthquakes.

There are about 6000 earthquakes recorded as of today, with several thousands of seismograms
available for each earthquake. About 50 iterations of a long processing pipeline for each earthquake
will be needed to achieve a highly accurate subsurface model of the whole Earth. In the pipeline,
thousands of seismograms (both observed and synthetic ones) are processed and produced at each step
for each earth quake while several earth quakes are processed concurrently. Observational data is
provided by the IRIS organization but it has to be processed (cleaned and aligned) for the
comparative analysis with the synthetic datasets. The current standard data format is a separate
binary file for each seismogram, resulting in millions of small input files and multiplies of that
number of files for intermediate synthetic data. Even the largest parallel file systems cannot
provide access to these many files with reasonable performance. Therefore, a new seismography data
format has been developed that stores all seismograms of an earth quake in a single ADIOS file, which
provides efficient access to all of them while reducing the total number of files by several orders
of magnitudes. This change alone allows for executing the processing pipeline on the Titan
supercomputer at the Oak Ridge Leadership Facility, using the allocated time efficiently, i.e., not
by mostly waiting on reading and writing the small files.
 
\vspace{10pt}


\noindent{\bf Climate research.}
Climate modeling applications address one of the most pressing issues faced by our society. Numerous
surveys show increasing trends in the average sea-surface and air temperatures. A promising method
of predicting the future effects of climate change is to run large-scale simulations of the global
climate many years into the future. In such simulations, climatologists discretize our entire globe,
setting up differential equations to model Earth's climate. These climate simulations can generate
several terabytes, or even petabytes, of raw data. Such data-intensive applications present a
tremendous set of challenges for computer centers that strive to accommodate, maintain, transfer,
and preserve data.

Large-scale computing centers that host such applications cannot afford to frequently stall on data
accesses and lose precious computing cycles. The I/O performance of scientific applications such as
climate models has a great impact on the completion of scientific simulation and post-mortem data
visualization and analysis. However, many applications have complex data characteristics that are
not well supported by existing parallel I/O libraries. For example, applications may generate large
number of small variables. While the data is large in total, each process only holds very small
amount of data for each variable. An example is one of the NASAs climate and weather models named
GEOS-5 (the Goddard Earth Observing System Model). A GEOS-5 simulation at a coarse resolution of
half-degree generates only 3.12GB data but which consists of 185 2-D variables and 80 3-D variables
at a time. The number of time steps can be configured to the order of thousands. It is challenging
to provide good I/O speed for both writing and reading where large amount of small I/O requests are
expected. Moreover, it is also quite common for data post-processing to examine data along time
dimension, e.g. observing the change of temperature within one hour. However, storage data layouts
currently common in use do not provide a good support for such access pattern.

An I/O technique that can significantly enhance the spatial and temporal resolution of these climate
and weather models can systematically lift the capability of climatologists and meteorologists in
addressing critical climate issues. An I/O method of ADIOS was specifically designed to address
these problems~\cite{ADIOS:Tian:msst13}. It leverages the spatial and temporal relationships between
variable data. Spatially it aggregates and merges data chunks of the same time steps so that fewer
processes write larger blocks. Moreover, the same variable across different time steps are merged
together with time as a new dimension, which again reduces the amount of I/O request for both
writing and reading. This strategy has provided two orders of magnitude of speedup compared to
simple writing (and reading) strategies.




%
%
\subsection{Code coupling}

ADIOS/DataSpaces~\cite{ADIOS:Docan:cluster12} is a scalable data-staging substrate that supports advanced coordination and interaction services for extreme scale coupled simulation workflows. It provides the abstractions and mechanisms to support flexible and dynamic inter-application coupling and interactions at runtime, and supports asynchronous data insertion and retrieval to/from a staging area composed of a set of cores on application nodes and dedicated staging nodes. For example, in case of a simulation-visualization workflow, the simulation can output data to the staging resources at runtime using the DataSpaces put() operator, and the visualization processes can asynchronously fetch relevant portions of the data using the corresponding get() operator. Note that these operators are independent of the scale and the distribution of these interacting applications. The internal data management mechanisms in DataSpaces ensure the scalability of the distributed data storage and lookup mechanisms. The DataSpaces runtime builds on DART~\cite{ADIOS:Docan:ccpe10} enables direct low-overhead, high-throughput memory-to-memory communication between the interacting nodes using Remote Direct Memory Access (RDMA). 

DataSpaces implements a distributed query engine with simple and flexible query semantics to facilitate access to the data. An application component can query any data region from the global application domain, e.g., an individual, or a range of data points. Extending the query engine, DataSpaces builds higher-level data sharing and coupling abstractions to support complex and dynamic interaction and coupling patterns. Applications can query data regions on-demand, register to obtain continuous notifications of data availability, implement the publisher-subscriber programming paradigm, share and redistribute data in a decoupled manner, and move analytics code to the data.

DataSpaces consists of two main components: a client module used by the ADIOS/DataSpaces method, and a DataSpaces server module. The client library is light-weight and provides an asynchronous I/O API. It is integrated with ADIOS and the functionality is exposed through the ADIOS write/read semantics, which bundles the variables of the same output together, so that reading applications can access a set of variables of an output step consistently just like when reading from files.

ADIOS/DataSpaces is currently used in production in coupled scientific simulation workflows on large-scale supercomputers. For example, as part of the fusion simulation framework (Figure~\ref{part3-ch6-adios:fig:coupling}), DataSpaces enables memory-to-memory coupling between the gyrokinetic PIC edge simulation code XGC0, and the MHD code M3D-OMP~\cite{ADIOS:Docan:ccgrid10}. Similarly, as part of the turbulent combustion workflow DataSpaces enables data coupling between the direct numerical simulations (DNS) code S3D and the data analytics pipeline~\cite{ADIOS:Bennett:SC12}.

\begin{figure}[h!]
\centering
\myIfColor
{
% use color version
\includegraphics[width=0.7\textwidth]{Chapters/part3-ch6-adios/figs/coupling.png}
}
{
% use B&W version
\includegraphics[width=0.7\textwidth]{Chapters/part3-ch6-adios/figs/coupling.png}
}
\caption[]
{Coupled Fusion Simulation Workflow using ADIOS and DataSpaces.
}
\label{part3-ch6-adios:fig:coupling}
\end{figure}


%
%
\subsection{Visualization}

A visualization pipeline is shown in Figure~\ref{part3-ch6-adios:fig:intransitviz}. Pixie3D \cite{ADIOS:Chacon:2002} is a  magneto-hydrodynamic simulation for fusion. This code's output cannot directly be used for visualization because of its internal geometries optimized for execution performance, not for human friendly representation. A separate code, Pixplot, has always been used to process the output on a small number of processes, transform and write the data and multiple meshes in Cartesian coordinates, feasible for visualization. The three components in the visualization pipeline, Pixie3D simulation, Pixplot transformation and ParaView parallel visualization server, represent a typical scenario for concurrent in~transit visualization. Besides the interactive visualization, a light sequential code, Pixmon, is used to extract 2D slices of the original output and to upload images of those slices to a web-accessible dashboard (eSiMon~\cite{ADIOS:Tchoua:cts12}), providing a continuous and remote monitoring of the simulation.


\begin{figure}[h!]
\centering
\myIfColor
{
% use color version
\includegraphics[width=0.5\textwidth]{Chapters/part3-ch6-adios/figs/intransitviz.png}
}
{
% use B&W version
\includegraphics[width=0.5\textwidth]{Chapters/part3-ch6-adios/figs/intransitviz-bw.png}
}
\caption[]
{In~transit analysis and visualization pipeline using ADIOS.
}
\label{part3-ch6-adios:fig:intransitviz}
\end{figure}

The application codes are written as if writing data to files and reading data from files, processing one step at a time and only advancing forward in time. ADIOS allows for a seamless switch from file-based post-processing to online processing helped with memory-to-memory data transfers. Staging methods that transfer data directly from the memory of the producer into the memory of the consumer are best for coupling and for straight analytical pipelines (e.g. DIMES and FlexPath methods of ADIOS). The DataSpaces model~\cite{ADIOS:Docan:cluster12}, however, provides a virtual shared multi-dimensional space using a set of separate nodes as a ``staging area,". Thus multiple, parallel applications can simultaneously read a multi-dimensional array, with an arbitrary decomposition. More importantly for interactive visualization, DataSpaces can hold as many output steps as the allocated memory can hold and can serve a reader with data the producer long has forgotten about. Moreover, DataSpaces provides fault isolation for the application. Failures downstream in a process pipeline does not propagate to the application. More details on using ADIOS and DataSpaces for in-transit visualization can be found in the book on High Performance Visualization~\cite{ADIOS:Bethel-Childs-Hansen:HPV:2012}.


%
%
\subsection{Data reduction}

Application of data reduction techniques with ADIOS have shown to be effective in reducing the
bottleneck on I/O during simulation writes. These techniques fit well with the minimal communication
principle employed by ADIOS, where each process handles its own I/O. By applying compression
routines locally on every process, encoding costs are effectively minimized. This also enables
compression techniques to take advantage of similarity in data values within each process, typically
seen with spatio-temporal scientific datasets. For example, ISOBAR \cite{ADIOS:Schendel2012b}, an in situ
lossless compression routine specific for scientific data, demonstrated up to a 46\% reduction in
storage on datasets from simulations spanning various domains such as combustion (S3D), plasma
(XGC1, GTS) and astrophysics (FLASH). This technique coupled with ADIOS and with the addition of
interleaving allowed throughput gains proportional to the degree of data reduction.

While checkpoint/restart data such as particle data must be compressed losslessly, datasets that are
used for analysis and visualization, such as field data, can be compressed in a lossy fashion.
Unlike lossless compression, lossy compression techniques such as wavelets-based compression,
ISABELA \cite{ADIOS:Lakshminarasimhan2013a}, etc. provide a multi-fold reduction in storage sizes, trading
precision for reduced storage. Depending on the sensitivity of the analysis routines to errors
introduced by the compression process, the end-users can change the level of accuracy desired in the
configuration file used by ADIOS. The configuration file can also instruct ADIOS to employ different
compression routines for different variables and output groups (checkpoint/restart or analysis),
without having to change the application code.


%
%
\subsection{Deployment}
ADIOS is an open source software with a BSD license. It has been installed as central software and is supported organizationally by the Oak Ridge Leadership Facility, by the iVEC organization in Western Australia for the purpose of supporting the Square-Kilometer Array project, by the High Level Support Team (HLST) supporting EFDA (European Fusion Development Agreement) sites, and by the ERDC center of the US. Army Corps of Engineers. At other locations, application scientists install ADIOS themselves for their own application. Many of those receive direct guidance from the ADIOS team to utilize their target system optimally.



%\index{Data Formats}
%\index{I/O middleware}

%
% Conclusion
%
\section{Conclusion}
{\color {red}Summary of experiences with the library. Future challenges and directions.}

%This is a reference to a book chapter~\ref{ch0:book-intro}.

\putbib[Chapters/part3-ch6-adios/part3-ch6-adios]
